\documentclass[a4paper,12pt,fleqn]{article}
\setlength{\parindent}{0em}
\usepackage{amsmath}
\usepackage{fancyhdr}
\usepackage{siunitx}
\usepackage{enumitem}
\usepackage{amsmath}
\usepackage{graphicx}
\usepackage{tikz}
\usepackage{import}
\usepackage{comment}
\usepackage{setspace}

% Unit definitions %%%%%%%%%%%%%%%%%%%%%%%%%%%%%%%%%%

\DeclareSIUnit\kilowatthour{kWh}
\DeclareSIUnit\kilowattpeak{kW_P}
\DeclareSIUnit\kVA{kVA}
\DeclareSIUnit\kVAR{kVAR}
\DeclareSIUnit\year{y}
\DeclareSIUnit\north{N}
\DeclareSIUnit\south{S}
\DeclareSIUnit\second{s}

\begin {document}

\textbf{Question One}

\begin{enumerate}[label=\alph*)]
\begin{doublespace}
\item $\mathrm{HLC (conduction) }=q_C=\sum{U_iA_i}=0.2\times 100+0.2\times 50+0.25\times 80 +1.0 \times 20=\SI{100}{\watt\per\kelvin}$
\item $\mathrm{HLC (infiltration) }=q_A=\frac{nV}{3}=\frac{0.2\times 240}{3}=\SI{16}{\watt\per\kelvin}$ using $n\approx\frac{n_{50}}{20}$
\item $\mathrm{HLP}=(q_C+q_A)/\mathrm{TFA}=\frac{100+16}{50}=\SI{2.32}{\watt\per\kelvin\per\metre\squared}$
\item $\mathrm{Annual\ Heat\ load}=(q_C+q_A)\times DD \ \frac{24}{1000}=116\times 1800 \times \frac{24}{1000}=\SI{5011}{\kilowatthour}$
\item Factors causing (d) to be too high: Use of intermittent heating; solar and other gains have been ignored.
Factor causing (d) to be too low: houses not built to specification so that $q_C$ and $q_A$ are higher than calculated here.
Actual degree days could differ from 20 year average.
\end{doublespace}
\item 
\begin{align*}
R_T&=\sum{x_i/k_i}=\frac{0.05}{0.7}+\frac{y}{0.055}=\frac{1}{U}=5\\
\intertext{Hence}
\frac{y}{0.055}&=5-\frac{0.05}{0.7}=4.93\\
\intertext{so}
y&=0.055\times 4.93 = \SI{0.27}{\metre}
\end{align*}
\item Heat conduction across the cavity is by conduction, convection and radiation. The radiation contribution is proportional
to the emissivity ($0<\epsilon <1$) of the surfaces.
Low emissivity surfaces such as shiny foil backing can significantly reduce the radiation conduction
\item Credit: Ties bridge(are in parallel with) the air gap. They likely have a high thermal conductivity, however their total cross-sectional
area will be much lower than that of the air in the gap. Thus their overall thermal-short effect is (designed to be) low. 
Credit attempt at demonstrating this by calculation
\item Credit attempts to calculate mass per $m^2$ of each component, using $m=\rho At$ where $A$ is the are fraction and $t$ is the thickness.
\begin{table}[ht]
\caption{Straw bale wall} % title of Table
\centering % used for centering table
\begin{tabular}{c c c c c c c} % centered columns (4 columns)
\hline\hline %inserts double horizontal lines
Component & Density & Area Fraction & Thickness & Mass & Emb. Energy / kg & Total Emb. Energy.\\  [0.5ex] %heading
\hline % inserts single horizontal line
Lime & 1600 & 1 & .05 & 80 &  1.03 & 82.4\\% inserting body of the table
Straw & 120 & 1 & 0.27 & 32.4 & 0.24 & 7.8\\
\hline
 & & & & &Total &\SI{90.2}{\mega\joule\per\metre\squared}\\ [1ex] % [1ex] adds vertical space
\hline %inserts single line
\end{tabular}
\label{table:q1} % is used to refer this table in the text
\end{table}

\end{enumerate}

\end{document}
