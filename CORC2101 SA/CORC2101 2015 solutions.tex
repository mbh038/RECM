\documentclass[a4paper,12pt,fleqn]{article}
\setlength{\parindent}{0em}
\usepackage{amsmath}
\usepackage{fancyhdr}
\usepackage{siunitx}
\usepackage{enumitem}
\usepackage{amsmath}
\usepackage{graphicx}
\usepackage{tikz}
\usepackage{import}
\usepackage{comment}
\usepackage{setspace}

% Unit definitions %%%%%%%%%%%%%%%%%%%%%%%%%%%%%%%%%%

\DeclareSIUnit\kilowatthour{kWh}
\DeclareSIUnit\kilowattpeak{kW_P}
\DeclareSIUnit\kVA{kVA}
\DeclareSIUnit\kVAR{kVAR}
\DeclareSIUnit\year{y}
\DeclareSIUnit\north{N}
\DeclareSIUnit\south{S}
\DeclareSIUnit\second{s}

\begin {document}

\textbf{Question One}

\begin{enumerate}[label=\alph*)]
\begin{doublespace}
\item $\mathrm{HLC (conduction) }=q_C=\sum{U_iA_i}=0.2\times 100+0.2\times 50+0.25\times 80 +1.0 \times 20=\SI{100}{\watt\per\kelvin}$
\item $\mathrm{HLC (infiltration) }=q_A=\frac{nV}{3}=\frac{0.2\times 240}{3}=\SI{16}{\watt\per\kelvin}$ using $n\approx\frac{n_{50}}{20}$
\item $\mathrm{HLP}=(q_C+q_A)/\mathrm{TFA}=\frac{100+16}{50}=\SI{2.32}{\watt\per\kelvin\per\metre\squared}$
\item $\mathrm{Annual\ Heat\ load}=(q_C+q_A)\times DD \ \frac{24}{1000}=116\times 1800 \times \frac{24}{1000}=\SI{5011}{\kilowatthour}$
\item Factors causing (d) to be too high: Use of intermittent heating; solar and other gains have been ignored.
Factor causing (d) to be too low: houses not built to specification so that $q_C$ and $q_A$ are higher than calculated here.
Actual degree days could differ from 20 year average.
\end{doublespace}
\item 
\begin{align*}
R_T&=\sum{x_i/k_i}=\frac{0.05}{0.7}+\frac{y}{0.055}=\frac{1}{U}=5\\
\intertext{Hence}
\frac{y}{0.055}&=5-\frac{0.05}{0.7}=4.93\\
\intertext{so}
y&=0.055\times 4.93 = \SI{0.27}{\metre}
\end{align*}
\item Heat conduction across the cavity is by conduction, convection and radiation. The radiation contribution is proportional
to the emissivity ($0<\epsilon <1$) of the surfaces.
Low emissivity surfaces such as shiny foil backing can significantly reduce the radiation conduction
\item Credit: Ties bridge(are in parallel with) the air gap. They likely have a high thermal conductivity, however their total cross-sectional
area will be much lower than that of the air in the gap. Thus their overall thermal-short effect is (designed to be) low. 
Credit attempt at demonstrating this by calculation
\item Credit attempts to calculate mass per $m^2$ of each component, using $m=\rho At$ where $A$ is the are fraction and $t$ is the thickness.
\begin{table}[ht]
\caption{Straw bale wall} % title of Table
\centering % used for centering table
\begin{tabular}{c c c c c c c} % centered columns (4 columns)
\hline\hline %inserts double horizontal lines
Component & Density & AF & t & m & EE / kg & EE.\\  [0.5ex] %heading
\hline % inserts single horizontal line
Lime & 1600 & 1 & .05 & 80 &  1.03 & 82.4\\% inserting body of the table
Straw & 120 & 1 & 0.27 & 32.4 & 0.24 & 7.8\\
\hline
 & & & & &Total &\SI{90.2}{\mega\joule\per\metre\squared}\\ [1ex] % [1ex] adds vertical space
\hline %inserts single line
\end{tabular}
\label{table:straw} % is used to refer this table in the text
\end{table}
\item Accept any reasonable choices for materials, dimensions etc.
\begin{table}[ht]
\caption{Concrete block cavity wall} % title of Table
\centering % used for centering table
\begin{tabular}{c c c c c c c} % centered columns (4 columns)
\hline\hline %inserts double horizontal lines
Component & Density & AF & t & m & EE / kg & EE.\\  [0.5ex] %heading
\hline % inserts single horizontal line
Concrete & 2300 & .93 & 0.2 & 428 &  0.7 & 299.4\\% inserting body of the table
Mortar & 1900 & 0.07 & 0.2 & 26.6 & 0.97 & 25.8\\
Celotex & 35 & 1 & 0.1 & 3.5 & 101 & 353.5\\
Plasterboard & 700 & 1.0 & 0.0125 & 8.75 & 6.75 & 59.1\\
Plaster dabs & 600& 0.2 & 0.025 & 3 & 1.8 & 5.4\\
Render+skim & 1900 & 1 & .013 & 24.7 & 0.97 & 23.96\\
\hline
 & & & & &Total &\SI{767}{\mega\joule\per\metre\squared}\\ [1ex] % [1ex] adds vertical space
\hline %inserts single line
\end{tabular}
\label{table:conc} % is used to refer this table in the text
\end{table}
\item For concrete wall, assume travel distance = 200 km - likely the case for the cement fraction. Total mass of walls $=200\times .50 = \SI{100}{\tonne}$, transport energy $= 100 \times 200 = \SI{20000}{\kilowatthour}$. This is about equal to the cradle to gate embodied energy of the materials, so local sourcing of concrete will make a significant impact on total EE.
Similar arguments apply to straw wall - credit attempt at quantitative argument, also attempt to assess relative importance of transport energy in whole-life energy of house.
\item $q_C \gg q_A$ hence more effort to improve insulation rather than infiltration will pay greater dividends. Increased, managed solar gain, provided it does not lead to a cooling energy cost, could also help.
\item Straw bales reduce embodied energy of walls by $\SI{680}{\mega\joule\per\metre\squared}=\SI{190}{\kilowatthour\per\metre\squared}$, hence by $\SI{19,000}{\kilowatthour}$ in total. This represents less than 4 years worth of in-use energy. However if that energy were reduced still further, this reduction in embodied energy becomes a greater and greater fraction, eventually the dominant fraction, of the whole-life energy.
\end{enumerate}


\textbf{Question Two}
\begin{enumerate}[label=\alph*)]
\item Provides sufficient accessible heat capacity, ideally within the thermal envelope of the building, matched to solar gains, to moderate interior temperature swings in response to external swings.
\item Because mode of transfer of heat to the wall is by convection and conduction, which relies on a temperature gradient being set up  across a boundary, with the surface of the wall at a lower temperature than the bulk temperature of the air.
\item Average temperature rises after 10 hours are $T_1 \approx \SI{4}{\celsius}$ for skin 1 and $T_2 \approx \SI{1}{\celsius}$ for skin 2.
\begin{align*}
E_1 &=mc\Delta T_1\\
&=\rho V c \Delta T_1\\
&=2300\times 0.15\times 840\times \Delta T_1\\
&=\SI{2.9e5}{joule\per\metre\squared\per\celsius}\cdot \Delta T_1\\
&=\SI{1.2}{\mega\joule\per\metre\squared}=\SI{0.32}{\kilowatthour\per\metre\squared}
\end{align*}
and
\begin{align*}
E_2 &=mc\Delta T_2\\
&=\SI{2.9e5}{joule\per\metre\squared\per\celsius}\cdot \Delta T_2\\
&=\SI{0.29}{\mega\joule\per\metre\squared}=\SI{0.08}{\kilowatthour\per\metre\squared}
\end{align*}
\item The second skin stores one quarter the thermal energy after 10 hours, for double the additional mass, thickness, embodied energy and cost.
\item Assuming $\Delta T=\SI{4}{\celsius}$, then after 10 hours the stored thermal energy $\approx 65\times 0.32=\SI{21}{\kilowatthour}$
\item The maximum solar gain per day is $20-\SI{25}{\kilowatthour\per\day}$ this is about equal to what can be absorbed by the skin of blocks in one day, albeit with temperature rise of \SI{4}{\celsius}. If this is too much, then the solar gains needs to be controlled in the summer months, perhaps by use of shades, or by reducing window size, or by using an intermediate space such as a conservatory to directly receive the solar gain.
\item 
\end{enumerate}
\end{document}
